\documentclass[10pt]{article}

\usepackage{clj-grammar}
\usepackage{amsmath}
%\usepackage{enumerate}
%\usepackage{fancyvrb}
%\usepackage{amsfonts,graphicx}
%\usepackage[usenames, dvipsnames]{color}
%\usepackage{tikz}

%\setlength{\parindent}{4em}
%\setlength{\parskip}{1em}
%\renewcommand{\baselinestretch}{1.5}


\usepackage[utf8]{inputenc}
\usepackage[english]{babel}
\usepackage{enumitem}

\begin{document}
\title{Qualifying Exam}
\author{Ambrose Bonnaire-Sergeant (0003410123)}

\maketitle

% Below are the three questions you should answer for your qualifying
% exam. You have three months to complete a written report answering
% these three questions.

\section*{Question 1}

\begin{verbatim}
Analyze the space and time complexity of your approach to dynamic
tracing & subsequent type inference for Typed Clojure.  Are you able
to bound space use at all by reducing traces as they are collected?
Please analyze a related system, Daikon
(https://plse.cs.washington.edu/daikon/), along the same lines.  How
expressive are Daikons invariants compared to yours?  How much space
and runtime overhead does it impose?  If Daikon's inferred invariants
were to become part of a (refinement) type system, how powerful would
it need to be?
\end{verbatim}

\subsection*{Space/time complexity of dynamic tracing}

\subsection*{Space/time complexity of type inference}

\subsection*{Can space use be bounded by reducing traces as collected?}

\subsection*{Daikon's expressivity}

\subsection*{Space/time overhead of Daikon}

\subsection*{How to type check Daikon's invariants}

% Notes:
%  Java implementation:
%    Chicory does instrumentation on JVM bytecode
%     instrument_all_methods: https://github.com/codespecs/daikon/blob/master/java/daikon/chicory/Instrument.java#L409

\section*{Question 2}

\begin{verbatim}
Examine the use of Clojure's core.spec contract system in several
real world code bases. Look at what features are used, and how precise
specifications are. Analyze how specifications address the lack of
higher-order contracts by looking at the frequency of higher-order function
contracts vs higher-order functions that omit specifications of higher-order
arguments or results.
\end{verbatim}

\subsection*{What spec features are used in real systems}

\subsection*{How precise are spec annotations in practice?}

\subsection*{How frequently are higher-order functions annotated with higher-order specs? Why?}

\section*{Question 3}

\begin{verbatim}
Write a formal model of Clojure with core.spec, and implement it in
PLT Redex. Formulate a consistency property between contracted and
uncontracted execution, and test it in redex.
\end{verbatim}

\begin{figure*}
$$
\begin{altgrammar}
  \expd{}, \e{} &::=& \x{}
                      \alt \v{} 
                      \alt {\comb {\e{}} {\e{}}} 
                      \alt {\abs {\x{}} {\t{}} {\e{}}}
                      \alt {\ifexp {\e{}} {\e{}} {\e{}}}
                      \alt {\doexp {\e{}} {\e{}}}
                &\mbox{Expressions} \\
  \v{} &::=&          \singletonmeta{}
                      \alt {\emptymap{}}
                      \alt {\num{}}
                      \alt \mapval{}
                      \alt {\closure {\openv{}} {\abs {\x{}} {\t{}} {\e{}}}}
                &\mbox{Values} \\
  \mapval{} &::=&  {\curlymapvaloverright{\v{}}{\v{}}}
                &\mbox{Map Values} \\
  \HashMapExp{}                &::=& \hmapexpressionsyntax{}
\end{altgrammar}
$$
\caption{Syntax of Terms and Specs}
\end{figure*}


\end{document}
